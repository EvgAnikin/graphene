\documentclass{article}

\usepackage{amsmath}
\usepackage[english, russian]{babel}
\usepackage[utf8]{inputenc}
\usepackage[T2A, T1]{fontenc}
\usepackage{graphicx}

\title{Краевые состояния в модели Кейна--Меле}
\author{Евгений Аникин}

\begin{document}
	\maketitle
	В предыдущем листке была получена формула для энергии в зависимости от квазиимпульса:
	\begin{multline}
		\epsilon_p^2 = \left(4t_2\sin{\frac{\sqrt{3}p_ya}{2}} \cos{\frac{3p_xa}{2}} - 
				4t_2\sin{\frac{\sqrt{3}p_ya}{2}} \cos{\frac{\sqrt{3}p_ya}{2}} +
				\frac{t^2}{2t_2} \cot{\frac{\sqrt{3}p_ya}{2}}\right)^2\\
				- \frac{t^4}{4t_2^2}\cot^2{\frac{\sqrt{3}p_ya}{2}}
				+ t^2\left(1 + 8\cos^2{\frac{\sqrt{3}p_ya}{2}}\right) = \\
				= \left(a\cos{\frac{3p_xa}{2}} + b\right)^2 + c
	\end{multline}
	Матрица гамильтониана --- 
	\begin{equation}
		\left(
		\begin{matrix}
			\xi && \eta e^{i\phi} \\
			\eta e^{-i\phi} && -\xi
		\end{matrix}
		\right),
	\end{equation}
	\begin{equation}
		\xi = 2t_2 (\sin{px} - \sin{py} - \sin{p(x-y)}) 
	\end{equation}
	\begin{equation}
		\eta e^{i\phi} = t(1 + e^{-ipy} + e^{ip(x-y)}) 
	\end{equation}
	Это значит, что функции Грина в импульсном представлении ---
	\begin{equation}
		G^R_0(\omega,A,A, \vec{p}) = \frac{\omega + \xi}{\omega^2 - \epsilon_p^2 + i\delta}
	\end{equation}
	\begin{equation}
		G^R_0(\omega,B,B, \vec{p}) = \frac{\omega - \xi}{\omega^2 - \epsilon_p^2 + i\delta}
	\end{equation}
	\begin{equation}
		G^R_0(\omega,A,B, \vec{p}) = \frac{\eta e^{-i\phi}}{\omega^2 - \epsilon_p^2 + i\delta}
	\end{equation}
	\begin{equation}
		G^R_0(\omega,B,A, \vec{p}) = G^R_0(\omega,A,B, -\vec{p})
	\end{equation}
	Теперь перейдём к рассмотрению границы цепочки. Пусть граница идёт вдоль вектора $\vec{y}$.
	Тогда уравнение Дайсона распадётся на независиме уравнения на функции $G(m,m',p_y)$. Границу
	можно реализовать, введя бесконечные добавки к энергии для атомов обоих типов вдоль
	одной линии.
	\begin{multline}
		G^R(m,s,m',s',p_y) = G^R_0(m,s,m',s'p_y)  \\
			+\Delta E G^R_0(m,s,0,A,p_y)G^R(0,A,m',s',p_y)  \\
						+\Delta E G^R_0(m,s,0,B,p_y)G^R(0,B,m',s',p_y) 
	\end{multline}
	Для бесконечного $\Delta E$ уравнение на связанные состояния ---
	\begin{equation}
		\operatorname{det} 
		\left(\begin{matrix}
			G^R_0(0,A,0,A,p_y) & G^R_0(0,A,0,B,p_y) \\
			G^R_0(0,B,0,A,p_y) & G^R_0(0,B,0,B,p_y) 
		\end{matrix}\right) = 0
	\end{equation}
	Таким образом, остаётся вычислить функции Грина, входящие в детеримнант. Вычисляются они 
	по таким формулам:
	\begin{equation}
		G_0^R(m,s,m',s',p_y) = 
			\nu_x\int \frac{dp_x}{2\pi} e^{i\vec{p}\vec{x}(m-m')} G^0_R(s,s',p_x,p_y)
	\end{equation}
	Этот интеграл равен сумме по полюсам в верхней полуплоскости, которые определяются уравнением
	\begin{equation}
		\omega^2 = (a\cos{k} + b)^2 + c
	\end{equation}
\end{document}
